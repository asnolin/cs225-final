\documentclass{article}
\title{CS225 Final Project Proposal}
\author{Andrew Nolin \\ git@asnolin \and Jared Wiggett \\ git@jwiggett}
\date{\today}

\begin{document}
\maketitle
\section*{Simply Typed Lambda Calculus with Exceptions}
We propose to implement a semantics and type checker for the simply typed lambda calculus extended with exception terms (errors, try, exceptions). 


\paragraph{Base Language}
The base language we will be using is the symply typed lambda calculus from chapter 9.1 from the Types and Programming Languages textbook. This calculus contains terms (variables, abstractions and  applications), values, types and a typing context.
\paragraph{Extended Language}
We will extend the simply typed lambda calculus with types: 
\begin{enumerate}
	\item error. Which has the typing rule: T-Error
	\item trap errors. Which has the typing rule: T-Try
	\item raise exception. Which has the typing rule: T-Exn
	\item handle exception. Wich has the typing rule: T-Try
\end{enumerate}
\paragraph{Applications}

\paragraph{Project Goals}
For this project, we plan to complete:
\begin{enumerate}
\item A small-step semantics for the simply-typed lambda calculus extended with exceptions
\item A type checker for the simply-typed lambda calculus extended with exceptions

\paragraph{Expected Challenges}
We expect a challenge to be adding in helper functions to help implement the exception extensions to the language. In the homeworks, these were always provided for us and they seem difficult to incorporate. Also, implementing test suites for the language will be new, as these were also always provided for us. These may be tough simply because we are still not terribly strong OCaml programmers.

\paragraph{Timeline and Milestones}

\end{document}
